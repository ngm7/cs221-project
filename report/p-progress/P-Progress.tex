%----------------------------------------------------------------------------------------
%	PACKAGES AND OTHER DOCUMENT CONFIGURATIONS
%----------------------------------------------------------------------------------------

\documentclass{article}

%%%%%%%%%%%%%%%%%%%%%%%%%%%%%%%%%%%%%%%%%
% Lachaise Assignment
% Structure Specification File
% Version 1.0 (26/6/2018)
%
% This template originates from:
% http://www.LaTeXTemplates.com
%
% Authors:
% Marion Lachaise & François Févotte
% Vel (vel@LaTeXTemplates.com)
%
% License:
% CC BY-NC-SA 3.0 (http://creativecommons.org/licenses/by-nc-sa/3.0/)
% 
%%%%%%%%%%%%%%%%%%%%%%%%%%%%%%%%%%%%%%%%%

%----------------------------------------------------------------------------------------
%	PACKAGES AND OTHER DOCUMENT CONFIGURATIONS
%----------------------------------------------------------------------------------------

\usepackage{amsmath,amsfonts,stmaryrd,amssymb} % Math packages

\usepackage{enumerate} % Custom item numbers for enumerations

\usepackage[ruled]{algorithm2e} % Algorithms

\usepackage[framemethod=tikz]{mdframed} % Allows defining custom boxed/framed environments

\usepackage{listings} % File listings, with syntax highlighting
\lstset{
	basicstyle=\ttfamily, % Typeset listings in monospace font
}

%----------------------------------------------------------------------------------------
%	DOCUMENT MARGINS
%----------------------------------------------------------------------------------------

\usepackage{geometry} % Required for adjusting page dimensions and margins

\geometry{
	paper=a4paper, % Paper size, change to letterpaper for US letter size
	top=2.5cm, % Top margin
	bottom=3cm, % Bottom margin
	left=2.5cm, % Left margin
	right=2.5cm, % Right margin
	headheight=14pt, % Header height
	footskip=1.5cm, % Space from the bottom margin to the baseline of the footer
	headsep=1.2cm, % Space from the top margin to the baseline of the header
	%showframe, % Uncomment to show how the type block is set on the page
}

%----------------------------------------------------------------------------------------
%	FONTS
%----------------------------------------------------------------------------------------

\usepackage[utf8]{inputenc} % Required for inputting international characters
\usepackage[T1]{fontenc} % Output font encoding for international characters

\usepackage{XCharter} % Use the XCharter fonts

%----------------------------------------------------------------------------------------
%	COMMAND LINE ENVIRONMENT
%----------------------------------------------------------------------------------------

% Usage:
% \begin{commandline}
%	\begin{verbatim}
%		$ ls
%		
%		Applications	Desktop	...
%	\end{verbatim}
% \end{commandline}

\mdfdefinestyle{commandline}{
	leftmargin=10pt,
	rightmargin=10pt,
	innerleftmargin=15pt,
	middlelinecolor=black!50!white,
	middlelinewidth=2pt,
	frametitlerule=false,
	backgroundcolor=black!5!white,
	frametitle={Command Line},
	frametitlefont={\normalfont\sffamily\color{white}\hspace{-1em}},
	frametitlebackgroundcolor=black!50!white,
	nobreak,
}

% Define a custom environment for command-line snapshots
\newenvironment{commandline}{
	\medskip
	\begin{mdframed}[style=commandline]
}{
	\end{mdframed}
	\medskip
}

%----------------------------------------------------------------------------------------
%	FILE CONTENTS ENVIRONMENT
%----------------------------------------------------------------------------------------

% Usage:
% \begin{file}[optional filename, defaults to "File"]
%	File contents, for example, with a listings environment
% \end{file}

\mdfdefinestyle{file}{
	innertopmargin=1.6\baselineskip,
	innerbottommargin=0.8\baselineskip,
	topline=false, bottomline=false,
	leftline=false, rightline=false,
	leftmargin=2cm,
	rightmargin=2cm,
	singleextra={%
		\draw[fill=black!10!white](P)++(0,-1.2em)rectangle(P-|O);
		\node[anchor=north west]
		at(P-|O){\ttfamily\mdfilename};
		%
		\def\l{3em}
		\draw(O-|P)++(-\l,0)--++(\l,\l)--(P)--(P-|O)--(O)--cycle;
		\draw(O-|P)++(-\l,0)--++(0,\l)--++(\l,0);
	},
	nobreak,
}

% Define a custom environment for file contents
\newenvironment{file}[1][File]{ % Set the default filename to "File"
	\medskip
	\newcommand{\mdfilename}{#1}
	\begin{mdframed}[style=file]
}{
	\end{mdframed}
	\medskip
}

%----------------------------------------------------------------------------------------
%	NUMBERED QUESTIONS ENVIRONMENT
%----------------------------------------------------------------------------------------

% Usage:
% \begin{question}[optional title]
%	Question contents
% \end{question}

\mdfdefinestyle{question}{
	innertopmargin=1.2\baselineskip,
	innerbottommargin=0.8\baselineskip,
	roundcorner=5pt,
	nobreak,
	singleextra={%
		\draw(P-|O)node[xshift=1em,anchor=west,fill=white,draw,rounded corners=5pt]{%
		Question \theQuestion\questionTitle};
	},
}

\newcounter{Question} % Stores the current question number that gets iterated with each new question

% Define a custom environment for numbered questions
\newenvironment{question}[1][\unskip]{
	\bigskip
	\stepcounter{Question}
	\newcommand{\questionTitle}{~#1}
	\begin{mdframed}[style=question]
}{
	\end{mdframed}
	\medskip
}

%----------------------------------------------------------------------------------------
%	WARNING TEXT ENVIRONMENT
%----------------------------------------------------------------------------------------

% Usage:
% \begin{warn}[optional title, defaults to "Warning:"]
%	Contents
% \end{warn}

\mdfdefinestyle{warning}{
	topline=false, bottomline=false,
	leftline=false, rightline=false,
	nobreak,
	singleextra={%
		\draw(P-|O)++(-0.5em,0)node(tmp1){};
		\draw(P-|O)++(0.5em,0)node(tmp2){};
		\fill[black,rotate around={45:(P-|O)}](tmp1)rectangle(tmp2);
		\node at(P-|O){\color{white}\scriptsize\bf !};
		\draw[very thick](P-|O)++(0,-1em)--(O);%--(O-|P);
	}
}

% Define a custom environment for warning text
\newenvironment{warn}[1][Warning:]{ % Set the default warning to "Warning:"
	\medskip
	\begin{mdframed}[style=warning]
		\noindent{\textbf{#1}}
}{
	\end{mdframed}
}

%----------------------------------------------------------------------------------------
%	INFORMATION ENVIRONMENT
%----------------------------------------------------------------------------------------

% Usage:
% \begin{info}[optional title, defaults to "Info:"]
% 	contents
% 	\end{info}

\mdfdefinestyle{info}{%
	topline=false, bottomline=false,
	leftline=false, rightline=false,
	nobreak,
	singleextra={%
		\fill[black](P-|O)circle[radius=0.4em];
		\node at(P-|O){\color{white}\scriptsize\bf i};
		\draw[very thick](P-|O)++(0,-0.8em)--(O);%--(O-|P);
	}
}

% Define a custom environment for information
\newenvironment{info}[1][Info:]{ % Set the default title to "Info:"
	\medskip
	\begin{mdframed}[style=info]
		\noindent{\textbf{#1}}
}{
	\end{mdframed}
}
 % Include the file specifying the document structure and custom commands
\usepackage{graphicx}

%----------------------------------------------------------------------------------------
%	ASSIGNMENT INFORMATION
%----------------------------------------------------------------------------------------

\title{CS221: Project Progress Report} % Title of the assignment

\author{	\textbf{Nimisha Tandon}\\  \texttt{nimisha@stanford.edu} \\
		\textbf{Shaila Balaraddi}\\  \texttt{shailaab@stanford.edu} \\
		\textbf{Naman Muley}\\      \texttt{ngmuley@stanford.edu}}

\date{\today} % University, school and/or department name(s) and a date

%----------------------------------------------------------------------------------------

\begin{document}

\maketitle % Print the title

%----------------------------------------------------------------------------------------
%	INTRODUCTION
%----------------------------------------------------------------------------------------

\section{Introduction} % Unnumbered section

We are tackling the problem of identifying articles that could be potentially dangerous to a brand. Our approach has two steps. First, we take in a set of articles and classify them into categories. In the final application, this could be done by pulling articles from the internet daily. Once these categories are identified, in the second step, we create an impact score for the brand in question. The articles with top impact scores can be provided to the client as having high potential for impact to the brand.

This problem has a rich application of various algorithms for natural language processing. Since proposing this project, we have explored a few algorithms to create a better sense of what an impact score can be. We looked into utilizing unsupervised learning algorithms like GloVe and supervised algorithm implementations like Naive Bayes. These have given us better impact scores. Further more, it feels like we could perform some more fine tuning to come up with better models to bring a richer understanding of the \textit{impact score}

\maketitle % Print the title
%----------------------------------------------------------------------------------------
%	MODEL
%----------------------------------------------------------------------------------------
\section{Model} % Unnumbered section

\subsection {Classification}
In making progress for the project, we decided to give more weight to the impact score analysis section, since that's the more novel component of the solution approach. Our classification currently still uses a simple Stochastic Gradient Descent classifier. We found it's accuracy to be nearly 82 percent and thought we can improve upon that in the later parts of the project.

We still did look at using the Naive Bayes algorithm to perform classification, since that's what literature reported is it's primary use. It performed slightly better at classification than SGD but not by much.

\subsection {Impact Score}

\textit{Talk a bit about how GloVe and NB are good fits to calculate the impact score}
We explored and implemented a Naive Bayes classifier as an alternative classifier to build the model.

The Naive Bayes algorithm provides a very simple framework that builds on the probabilistic model based on training data.
The algorithm is based on a statistical classification method called Bayes Theorem.

In order to understand how naive Bayes classifiers work, we can represent Bayes rule as the following probabilistic model as follows:

\[Posteriorprobability = \frac{conditionalprobability * priorprobability}{evidence}\]
Bayes’ theorem forms the core of the whole concept of naive Bayes classification. The posterior probability, in the context of a classification problem, can be interpreted as: “What is the probability that a particular object belongs to class i given its observed feature values?” The answer to this question gives us a score of how much a document is relevant to a given topic.


Impact score needs to be a number that represents how relevant or impactful will a particular article be to the brand. Hence, we extended our previous approach of having a dictionary of words that are relevant to the brand and applying some NLP algorithms to come up with a score of how closely does the article's content relate to the dictionary relevant to the brand.

\maketitle
%----------------------------------------------------------------------------------------
%	ALGORITHM
%----------------------------------------------------------------------------------------
\section {Algorithm}

\subsubsection {GloVe}

\subsubsection {Naive Bayes}
The Naive Bayes algorithm:
Naive Bayes consists of multiple algorithms, like Multinomial Naive Bayes and Multi-variate Bernoulli Naive Bayes.

We found that the Multinomial Naive Bayes algorithm is well suited for our project. This is because this algorithm gives us the "term" frequency of a dictionary of words of a specific brand.
The term frequency is defined as the "number of times a given term appears in a given document". In practice the term frequency is often normalized by dividing the raw term frequency by the length of the document.

The term frequency can then be used to compute the max likelihood estimate based on the training data.

The algorithm to use for training using Naive Bayes:

function Train NaiveBayes(D, C)

for each category:

#Calculate the prior probability
num-docs = number of docs in D

num-category = number of docs in category

\[priorprobability[category] \leftarrow log \frac{N_{c}}{N_{doc}}\]

 V = vocabulary of D

 docs[category] <- append d for d subset of D in specified category

 for each word in V:

    Count(w,c) <- num of occurences of w in docs[category]

    \[priorprobability[category] \leftarrow log \frac{count(w,c) +1)}{\sum_{w} count(w,c) + |V|)}\]

return priorprobability

We can then use the priorprobability to test our data to get the maximum likleihood.

\maketitle
%---------i-------------------------------------------------------------------------------
%	IMPLEMENTATION
%----------------------------------------------------------------------------------------
\section{Implementation} % Unnumbered section

 \maketitle
%----------------------------------------------------------------------------------------
%	PRELIMINARY RESULTS
%----------------------------------------------------------------------------------------
\section{Preliminary Results} % Unnumbered section

\end{document}
