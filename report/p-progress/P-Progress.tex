%----------------------------------------------------------------------------------------
%	PACKAGES AND OTHER DOCUMENT CONFIGURATIONS
%----------------------------------------------------------------------------------------

\documentclass{article}

\input{structure.tex} % Include the file specifying the document structure and custom commands
\usepackage{graphicx}
\usepackage[ruled]{algorithm2e}
\usepackage{amsmath}
\SetCommentSty{mycommfont}
%----------------------------------------------------------------------------------------
%	ASSIGNMENT INFORMATION
%----------------------------------------------------------------------------------------

\title{CS221: Project Progress Report} % Title of the assignment

\author{	\textbf{Nimisha Tandon}\\  \texttt{nimisha@stanford.edu} \\
		\textbf{Shaila Balaraddi}\\  \texttt{shailaab@stanford.edu} \\
		\textbf{Naman Muley}\\      \texttt{ngmuley@stanford.edu}}

\date{\today} % University, school and/or department name(s) and a date

%----------------------------------------------------------------------------------------

\begin{document}

\maketitle % Print the title

%----------------------------------------------------------------------------------------
%	INTRODUCTION
%----------------------------------------------------------------------------------------

\section{Introduction} % Unnumbered section

We are tackling the problem of identifying articles that could be potentially dangerous to a brand. Our approach has two steps. First, we take in a set of articles and classify them into categories. In the final application, this could be done by pulling articles from the internet daily. Once these categories are identified, in the second step, we create an impact score for the brand in question. The articles with top impact scores can be provided to the client as having high potential for impact to the brand.

This problem has a rich application of various algorithms for natural language processing. Since proposing this project, we have explored a few algorithms to create a better sense of what an impact score can be. We looked into utilizing unsupervised learning algorithms like GloVe and supervised algorithm implementations like Naive Bayes. These have given us better impact scores. Further more, it feels like we could perform some more fine tuning to come up with better models to bring a richer understanding of the \textit{impact score}

\maketitle % Print the title
%----------------------------------------------------------------------------------------
%	MODEL
%----------------------------------------------------------------------------------------
\section{Methodology} % Unnumbered section
We extend our two step approach which we presented in the proposal. We have spent time exploring different algorithms for both these stages. We seem to have settled on to a methodology, which we explain in this section.

\begin{figure}
	\centering
 	 \includegraphics[width=0.6\linewidth]{impact_score.png}
	  \caption{Process Overview}
 	 \label{fig:Impact Potential}
\end{figure}

\subsection {Classification}
In making progress for the project, we decided to give more weight to the impact score analysis section, since that's the more novel component of the solution approach. Our classification currently still uses a simple Stochastic Gradient Descent classifier. We found it's accuracy to be nearly 82 percent and thought we can improve upon that in the later parts of the project.

We still did look at using the Naive Bayes algorithm to perform classification, since that's what literature reported is it's primary use. It performed slightly better at classification than SGD but not by much.

\subsection {Impact Score}

Impact score needs to be a number that represents how relevant or impactful will a particular article be to the brand. Hence, we extended our previous approach of having a vocabulary of words that are relevant to the brand and applying a combination GloVe analysis, TF-IDF and Naive Bayes to come up with a score of how closely does the article's content relate to the dictionary relevant to the brand.

\subsubsection {Input Brand Vocabulary}
It was possible to decipher a set of words that are deemed contextually more dominant and important for a category like \textit{guns} by various methodologies including TF-IDF itself. But we believe finding relevancy to a brand should allow some input bias from the brand. For example, a brand may way to find articles relevant to a subset of their very specific products and those may not be popularly present in the testing dataset. 

We allow a list of words to act as our vocabulary for finding articles with highest potential impact. For example, for a brand like NRA, we take the following set of words as a vocabulary:
\begin{center}
["guns", "shooting", "weapon", "nra", "handgun", "assault", "rifle", "america", "firearm"]
\end {center}

\subsubsection {Similar Words using GloVe learning}
A measure of how relevant an article is to a brand is to know how many of semantically or linguistically similar words occur in that article. For each of the words in the vocabulary, we get a list of \textit{similar words} using the GloVe learning algorithm [1]. We then use a simple TF algorithm to understand how commonly do these similar words occur in an article. 

\begin{center}
\begin{algorithm}
  \caption{Extend Vocabulary with Similar Words using Glove Model}
  \SetKwInOut{Input}{inputs}
  \SetKwInOut{Output}{output}
  \SetKwProg{GetGloveWords}{GetGloveWords}{}{}
  
 $Vocabulary \gets ["guns", "rifle", "weapon", "nra", "handgun", "firearm"] $\;
 $model \gets train\_glove(\textit{glove.6B.300d.w2vformat.tx}) $\;
 \GetGloveWords{$(Vocabulary, model)$} {
      $extended\_vocab \gets Vocabulary $\;
     \ForEach{word $w \in Vocabulary$}{%
       $similar\_words \gets model.get\_similar\_words($word=w, $topN=10) $\;
       $extended\_vocab.extend($similar\_words$) $\;
      }
    \KwRet{$extended\_vocab$}\;
  }
\end{algorithm}
\end{center}

Once the vocabulary is extended, we use this extended vocabulary to identify articles that have the most occurrences of these words. In the Proposal, we had used a similar algorithm to calculate our baseline but with the vocabulary as the raw input set of words. 

Our algorithm that performs this to come up with an impact score for each article in the category is given below. The score dictionary contains the impact score calculated for each article in the set of articles. We will combine this score with another score we calculate using the TF-IDF methods.

\begin {center}
\begin {algorithm}[ht]
\SetNoFillComment
\caption{Calculate Impact Score for each article from a list of articles, given extended vocabulary}
\SetKwInOut{Input}{inputs}
\SetKwInOut{Output}{output}
\SetKwProg{CalculateVocabImpactScore}{CalculateVocabImpactScore}{}{}
\CalculateVocabImpactScore{$(V, D)$} {
    \Input{A list of words that form ExtendedVocabulary V; a list of articles D}
    \Output{A dictionary representing impact scores for all articles in D}
    $score \gets \emptyset $\;
    $RangeD \gets range(0, len(D)) $\;
    \ForEach {index $i \in RangeD$} {
    	score[i] = 0 \;
	freq = [] \;
	\tcc{iterate over all words in $i_{th}$ article to count frequency}
	\ForEach {word $w \in D[i]$} {
		freq[word] += 1 \;
	}
	\tcc{iterate over all words in V to build a score for article $i$}
	\ForEach {word $w \in V$} {
		score[i] += freq[w] \;
	}
    }
    score.Normalize() \;
    \KwRet{$score$}\;
}
\end {algorithm}
\end {center}

\subsubsection {Relevant words using TF-IDF}

Term Frequency-Inverse Document Frequency commonly known as TF-IDF can be used to find the frequency of vocabulary words in a document to perform sentiment analysis or extract relevancy from a document.

This algorihm has 2 algorithms working together.
\begin{enumerate}
\item {Term Frequency}
\[tf(t,d) =  = \frac{no of occurrences of term in doc }{total number of all words in document}\]


Before using this algorithm, we first do some preprocessing of the doc as follows:
\begin{itemize}
\item {Remove all stop words.(ex: in, the,are it)}
\item {Convert all words to lowercase.}
\end {itemize}
Now using the term frequency alone will end up giving us words that are not unique (for ex: repeated words, should not add up to the relevance of the document)

\item Inverse document frequency

 This gives us the uniqueness of a word:

 \[idf(t,d) =  = log(\frac{no of times the term appears }{no of documents containing the word})\]

The final step is to multiply the two together to get the TFIDF score which would give us the impact score.
\end{enumerate}


\maketitle
%---------i-------------------------------------------------------------------------------
%	IMPLEMENTATION
%----------------------------------------------------------------------------------------
\section{Implementation} % Unnumbered section
 \textit{Talk about how the models are trained, what data sets did we use for these models. how are the data sets relevant. Future work may involve using and creating better data sets}

 \maketitle
%----------------------------------------------------------------------------------------
%	PRELIMINARY RESULTS
%----------------------------------------------------------------------------------------
\section{Preliminary Results} % Unnumbered section
\textit{Show similar words, maybe a fancy representation of similar words and important words as found by GloVe and TF-IDF respectively}

\maketitle
%----------------------------------------------------------------------------------------
%	PRELIMINARY RESULTS
%----------------------------------------------------------------------------------------
\section {Future Work}
While we have some results obtained from the  2. Have the application fetch testing data from today's newsfeed
\end{document}
