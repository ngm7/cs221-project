%----------------------------------------------------------------------------------------
%	PACKAGES AND OTHER DOCUMENT CONFIGURATIONS
%----------------------------------------------------------------------------------------

\documentclass{article}

%%%%%%%%%%%%%%%%%%%%%%%%%%%%%%%%%%%%%%%%%
% Lachaise Assignment
% Structure Specification File
% Version 1.0 (26/6/2018)
%
% This template originates from:
% http://www.LaTeXTemplates.com
%
% Authors:
% Marion Lachaise & François Févotte
% Vel (vel@LaTeXTemplates.com)
%
% License:
% CC BY-NC-SA 3.0 (http://creativecommons.org/licenses/by-nc-sa/3.0/)
% 
%%%%%%%%%%%%%%%%%%%%%%%%%%%%%%%%%%%%%%%%%

%----------------------------------------------------------------------------------------
%	PACKAGES AND OTHER DOCUMENT CONFIGURATIONS
%----------------------------------------------------------------------------------------

\usepackage{amsmath,amsfonts,stmaryrd,amssymb} % Math packages

\usepackage{enumerate} % Custom item numbers for enumerations

\usepackage[ruled]{algorithm2e} % Algorithms

\usepackage[framemethod=tikz]{mdframed} % Allows defining custom boxed/framed environments

\usepackage{listings} % File listings, with syntax highlighting
\lstset{
	basicstyle=\ttfamily, % Typeset listings in monospace font
}

%----------------------------------------------------------------------------------------
%	DOCUMENT MARGINS
%----------------------------------------------------------------------------------------

\usepackage{geometry} % Required for adjusting page dimensions and margins

\geometry{
	paper=a4paper, % Paper size, change to letterpaper for US letter size
	top=2.5cm, % Top margin
	bottom=3cm, % Bottom margin
	left=2.5cm, % Left margin
	right=2.5cm, % Right margin
	headheight=14pt, % Header height
	footskip=1.5cm, % Space from the bottom margin to the baseline of the footer
	headsep=1.2cm, % Space from the top margin to the baseline of the header
	%showframe, % Uncomment to show how the type block is set on the page
}

%----------------------------------------------------------------------------------------
%	FONTS
%----------------------------------------------------------------------------------------

\usepackage[utf8]{inputenc} % Required for inputting international characters
\usepackage[T1]{fontenc} % Output font encoding for international characters

\usepackage{XCharter} % Use the XCharter fonts

%----------------------------------------------------------------------------------------
%	COMMAND LINE ENVIRONMENT
%----------------------------------------------------------------------------------------

% Usage:
% \begin{commandline}
%	\begin{verbatim}
%		$ ls
%		
%		Applications	Desktop	...
%	\end{verbatim}
% \end{commandline}

\mdfdefinestyle{commandline}{
	leftmargin=10pt,
	rightmargin=10pt,
	innerleftmargin=15pt,
	middlelinecolor=black!50!white,
	middlelinewidth=2pt,
	frametitlerule=false,
	backgroundcolor=black!5!white,
	frametitle={Command Line},
	frametitlefont={\normalfont\sffamily\color{white}\hspace{-1em}},
	frametitlebackgroundcolor=black!50!white,
	nobreak,
}

% Define a custom environment for command-line snapshots
\newenvironment{commandline}{
	\medskip
	\begin{mdframed}[style=commandline]
}{
	\end{mdframed}
	\medskip
}

%----------------------------------------------------------------------------------------
%	FILE CONTENTS ENVIRONMENT
%----------------------------------------------------------------------------------------

% Usage:
% \begin{file}[optional filename, defaults to "File"]
%	File contents, for example, with a listings environment
% \end{file}

\mdfdefinestyle{file}{
	innertopmargin=1.6\baselineskip,
	innerbottommargin=0.8\baselineskip,
	topline=false, bottomline=false,
	leftline=false, rightline=false,
	leftmargin=2cm,
	rightmargin=2cm,
	singleextra={%
		\draw[fill=black!10!white](P)++(0,-1.2em)rectangle(P-|O);
		\node[anchor=north west]
		at(P-|O){\ttfamily\mdfilename};
		%
		\def\l{3em}
		\draw(O-|P)++(-\l,0)--++(\l,\l)--(P)--(P-|O)--(O)--cycle;
		\draw(O-|P)++(-\l,0)--++(0,\l)--++(\l,0);
	},
	nobreak,
}

% Define a custom environment for file contents
\newenvironment{file}[1][File]{ % Set the default filename to "File"
	\medskip
	\newcommand{\mdfilename}{#1}
	\begin{mdframed}[style=file]
}{
	\end{mdframed}
	\medskip
}

%----------------------------------------------------------------------------------------
%	NUMBERED QUESTIONS ENVIRONMENT
%----------------------------------------------------------------------------------------

% Usage:
% \begin{question}[optional title]
%	Question contents
% \end{question}

\mdfdefinestyle{question}{
	innertopmargin=1.2\baselineskip,
	innerbottommargin=0.8\baselineskip,
	roundcorner=5pt,
	nobreak,
	singleextra={%
		\draw(P-|O)node[xshift=1em,anchor=west,fill=white,draw,rounded corners=5pt]{%
		Question \theQuestion\questionTitle};
	},
}

\newcounter{Question} % Stores the current question number that gets iterated with each new question

% Define a custom environment for numbered questions
\newenvironment{question}[1][\unskip]{
	\bigskip
	\stepcounter{Question}
	\newcommand{\questionTitle}{~#1}
	\begin{mdframed}[style=question]
}{
	\end{mdframed}
	\medskip
}

%----------------------------------------------------------------------------------------
%	WARNING TEXT ENVIRONMENT
%----------------------------------------------------------------------------------------

% Usage:
% \begin{warn}[optional title, defaults to "Warning:"]
%	Contents
% \end{warn}

\mdfdefinestyle{warning}{
	topline=false, bottomline=false,
	leftline=false, rightline=false,
	nobreak,
	singleextra={%
		\draw(P-|O)++(-0.5em,0)node(tmp1){};
		\draw(P-|O)++(0.5em,0)node(tmp2){};
		\fill[black,rotate around={45:(P-|O)}](tmp1)rectangle(tmp2);
		\node at(P-|O){\color{white}\scriptsize\bf !};
		\draw[very thick](P-|O)++(0,-1em)--(O);%--(O-|P);
	}
}

% Define a custom environment for warning text
\newenvironment{warn}[1][Warning:]{ % Set the default warning to "Warning:"
	\medskip
	\begin{mdframed}[style=warning]
		\noindent{\textbf{#1}}
}{
	\end{mdframed}
}

%----------------------------------------------------------------------------------------
%	INFORMATION ENVIRONMENT
%----------------------------------------------------------------------------------------

% Usage:
% \begin{info}[optional title, defaults to "Info:"]
% 	contents
% 	\end{info}

\mdfdefinestyle{info}{%
	topline=false, bottomline=false,
	leftline=false, rightline=false,
	nobreak,
	singleextra={%
		\fill[black](P-|O)circle[radius=0.4em];
		\node at(P-|O){\color{white}\scriptsize\bf i};
		\draw[very thick](P-|O)++(0,-0.8em)--(O);%--(O-|P);
	}
}

% Define a custom environment for information
\newenvironment{info}[1][Info:]{ % Set the default title to "Info:"
	\medskip
	\begin{mdframed}[style=info]
		\noindent{\textbf{#1}}
}{
	\end{mdframed}
}
 % Include the file specifying the document structure and custom commands

%----------------------------------------------------------------------------------------
%	ASSIGNMENT INFORMATION
%----------------------------------------------------------------------------------------

\title{CS221: Project Proposal} % Title of the assignment

\author{	\textbf{Nimisha Tandon}\\  \texttt{nimisha@stanford.edu} \\
		\textbf{Shaila Balaraddi}\\  \texttt{shailaab@stanford.edu} \\ 
		\textbf{Naman Muley}\\      \texttt{ngmuley@stanford.edu}}

\date{\today} % University, school and/or department name(s) and a date

%----------------------------------------------------------------------------------------

\begin{document}

\maketitle % Print the title

%----------------------------------------------------------------------------------------
%	INTRODUCTION
%----------------------------------------------------------------------------------------

\section*{Introduction} % Unnumbered section

Identifying information on the internet that could be potentially damaging to a brand is an important function of marketing and PR for a brand. News articles about the entire industry, a product ecosystem or the brand itself, could be relevant to a brand and impact it positively or negatively. A system that provides an impact score by analyzing news and other textual documents can provide useful leads for a brand to get ahead of a PR cycle. For example, articles in Mexican national newspapers about use of guns can be relevant to NRA as a brand in North America and will be helpful for it to shape it's policies.
\newline \newline
For the final project we will, first, classify an article into a news group. Once classified we will analyze the article to produce an impact score for a set of pre-defined parameters. These parameters give insight into how these articles will affect a brand. The impact score can be used to stack rank articles, when presented to the brand. 

\maketitle % Print the title

%----------------------------------------------------------------------------------------
%	SCOPE
%----------------------------------------------------------------------------------------
\section*{Project Scope} % Unnumbered section

This project can be broken down into two stages: a) classification and b) impact analysis. Both these operations will deploy ML techniques. Both these stages also require training effort. Each of the following stages will take a set of training data to train the algorithm. They will then take as input 

\subsection*{Classification into News Groups}

We will start by exploring to build a text classifier. To build this we will explore multiple techniques like RNN using LSTM, CNN and ML. Based on our findings we plan to either chose 1 classification model or create an ensemble of the 3 models and use that to classify our article. To name a few we would be exploring the below(and more) features to study each ones impact on the classification model. We will compare the output of the model looking at the  Precision , Recall and F1 scores on the validation set: 
	
\begin{itemize}
	\item Bag of words 
	\item Word2Vec using Glove vectors
	\item TF-IDF
	\item Part of Speech Tagging
	\item Averaging word vectors 
\end{itemize}

\subsection*{Impact Analysis}
Once the article is categorized, providing an impact score based on the classification is the next step. Creating a meaningful impact score is key and part of the challenge. Impact to a brand can be classified based on certain parameters that are important to a brand e.g. market share, sentiment of it's products, sales, research and development etc. Some of these parameters are specific to a category and some others are generally relevant to the brand.

Following are some parameters for a category like \textit{geography}:
\begin{itemize}
	\item \textit{country}: Is the geographical location talked about in this article one that is highly relevant to the brand?
	\item \textit{population}: Is this article relevant to a high population number or low?
\end{itemize}

Following are some general parameters that could be relevant to the brand:
\begin{itemize}
	\item \textit{Features}: are people talking about a particular aspect of your product or service?
	\item \textit{Sentiment}: is the article talking about the brand in positive or negative light
	\item \textit{Sales}: is the article talking price, sales or any other monetary factors relevant to the brand
\end{itemize}

Similar to classification, use of RNN or CNN or a combination of ML techniques can be used to build the impact score.
%----------------------------------------------------------------------------------------
%	ORACLE AND BASELINE
%----------------------------------------------------------------------------------------

\section*{Oracle and Baseline} % Unnumbered section%

The Oracle will be a manual classification of articles, and impact score based on the content and its relevance to the subject.
Following is a table of the results of the articles that have any impact to the subject of "guns".
We are interested in articles that reflect a true impact to guns, in the context of gun control, gun legislation, NRA and 2nd amaendment
The following list of articles are categorized and evaluated.

Article Name    Category    Impact score(scale of 1 to 10)
53297           Politics    10
53294           Politics    10
176846          Politics    0
crime_report    Police      0

As per the Oracle, the scores are dependent on the context of the domain in question.
Hence even though the article "crime_report" has words like shooting and homicide, it doesnt have any impact on the domain of "guns".

%----------------------------------------------------------------------------------------
%	CHALLENGES
%----------------------------------------------------------------------------------------

\section*{Challenges} % Unnumbered section

Following are some key challenges facing the project:
\begin{itemize}
	\item Build a system that understands impact score
	\item Build a classifier system that can compliment the impact score better
	\item Obtaining relevant data sets to train the system
\end{itemize}

 
%----------------------------------------------------------------------------------------
%	INFRASTRUCTURE
%----------------------------------------------------------------------------------------

\section*{Infrastructure} % Unnumbered section
The details of the infrastructure required for this project is detailed below. The "Application and Interface" section provides a basic information about
the core of the project (Ex: code and any api calls). Compute is used to specify the infrastructure used for the project. The section for "Dataset" will cover
the source, preprocessing and cleanup of data done so far.

\subsection*{Application and interfaces}
The project will be written in Python using Scikit-learn library as needed. \newline
The model will accept input from user on the category and will display the outputs as a stack rank of articles on the internet along with their potential impact scores. \newline
This application will interact with the model. \newline
The application will also be responsible for pulling articles from the internet in obtaining test data if required.

\subsection*{Compute}
The project will be run on GCP (Google Cloud platform) using the google cloud credits.

\subsection*{Dataset}
We will be using the “20 Newsgroup” data set from sklearn. The 20 Newsgroups data set is a collection of approximately 20,000 newsgroup documents, partitioned (nearly) evenly across 20 different newsgroups. \newline
To the best of our knowledge, it was originally collected by Ken Lang, probably for his Newsweeder: Learning to filter netnews paper, though he does not explicitly mention this collection.
The 20 newsgroups collection has become a popular data set for experiments in text applications of machine learning techniques, such as text classification and text clustering. This comes builtin with scikit learn. And can be used directly from there. \newline
We are also looking to use the bbc data set which consists of 2225 documents from the BBC news website corresponding to stories in five topical areas business, entertainment, politics, sport, tech.

The baseline is doing the data cleanup and preprocessing using the TFIDFVectorizer from Scikit-learn.
We also explored the Bagofwords algorithm before finalizing on the tfidf for training, as this has additional capability of cleaning up words that are too common across documents.
To provide impact analysis, we have manually built a vocabulary set to look for specific domain in the articles, for two categories.
TODO Examples of the data here:
Vocab for Guns:
guns, shooting, victim, mass, kill, murder, weapon, gun, nra, handgun, assault


\maketitle % Print the title

%----------------------------------------------------------------------------------------
%	SOCIAL IMPACT
%----------------------------------------------------------------------------------------

\section*{Infrastructure} % Unnumbered section

\section*{Project Prompt}

\textit{Define the input-output behavior of the system and the scope of the project. What is your evaluation metric for success? Collect some preliminary data, and give concrete examples of inputs and outputs. Implement a baseline and an oracle and discuss the gap. What are the challenges? Which topics (e.g., search, MDPs, etc.) might be able to address those challenges (at a high-level, since we haven't covered any techniques in detail at this point)? Search the Internet for similar projects and mention the related work. You should basically have all the infrastructure (e.g., building a simulator, cleaning data) completed to do something interesting by now.}
\end{document}
