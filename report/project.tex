%----------------------------------------------------------------------------------------
%	PACKAGES AND OTHER DOCUMENT CONFIGURATIONS
%----------------------------------------------------------------------------------------

\documentclass{article}

\input{structure.tex} % Include the file specifying the document structure and custom commands

%----------------------------------------------------------------------------------------
%	ASSIGNMENT INFORMATION
%----------------------------------------------------------------------------------------

\title{CS221: Project Proposal} % Title of the assignment

\author{	\textbf{Nimisha Tandon}\\  \texttt{nimisha@stanford.edu} \\
		\textbf{Shaila Balaraddi}\\  \texttt{shailaab@stanford.edu} \\ 
		\textbf{Naman Muley}\\      \texttt{ngmuley@stanford.edu}}

\date{\today} % University, school and/or department name(s) and a date

%----------------------------------------------------------------------------------------

\begin{document}

\maketitle % Print the title

%----------------------------------------------------------------------------------------
%	INTRODUCTION
%----------------------------------------------------------------------------------------

\section*{Introduction} % Unnumbered section

Identifying information on the internet that could be potentially damaging to a brand is an important function of marketing and PR for a brand. News articles about the entire industry, a product ecosystem or the brand itself, could be relevant to a brand and impact it positively or negatively. A system that provides an impact score by analyzing news and other textual documents can provide useful leads for a brand to get ahead of a PR cycle. For example, articles in Mexican national newspapers about use of guns can be relevant to NRA as a brand in North America and will be helpful for it to shape it's policies.
\newline \newline
For the final project we will, first, classify an article into a news group. Once classified we will analyze the article to produce an impact score for a set of pre-defined parameters. These parameters give insight into how these articles will affect a brand. The impact score can be used to stack rank articles, when presented to the brand. 

\maketitle % Print the title

%----------------------------------------------------------------------------------------
%	SCOPE
%----------------------------------------------------------------------------------------
\section*{Project Scope} % Unnumbered section

This project can be broken down into two stages: a) classification and b) impact analysis. Both these operations will deploy ML techniques. Both these stages also require training effort. Each of the following stages will take a set of training data to train the algorithm. They will then take as input 

\subsection*{Classification into News Groups}

We will start by exploring to build a text classifier. To build this we will explore multiple techniques like RNN using LSTM, CNN and ML. Based on our findings we plan to either chose 1 classification model or create an ensemble of the 3 models and use that to classify our article. To name a few we would be exploring the below(and more) features to study each ones impact on the classification model. We will compare the output of the model looking at the  Precision , Recall and F1 scores on the validation set: 
	
\begin{itemize}
	\item Bag of words 
	\item Word2Vec using Glove vectors
	\item TF-IDF
	\item Part of Speech Tagging
	\item Averaging word vectors 
\end{itemize}

\subsection*{Impact Analysis}
Once the article is categorized, providing an impact score based on the classification is the next step. Creating a meaningful impact score is key and part of the challenge. Impact to a brand can be classified based on certain parameters that are important to a brand e.g. market share, sentiment of it's products, sales, research and development etc. Some of these parameters are specific to a category and some others are generally relevant to the brand.

Following are some parameters for a category like \textit{geography}:
\begin{itemize}
	\item \textit{country}: Is the geographical location talked about in this article one that is highly relevant to the brand?
	\item \textit{population}: Is this article relevant to a high population number or low?
\end{itemize}

Following are some general parameters that could be relevant to the brand:
\begin{itemize}
	\item \textit{Features}: are people talking about a particular aspect of your product or service?
	\item \textit{Sentiment}: is the article talking about the brand in positive or negative light
	\item \textit{Sales}: is the article talking price, sales or any other monetary factors relevant to the brand
\end{itemize}

Similar to classification, use of RNN or CNN or a combination of ML techniques can be used to build the impact score.
%----------------------------------------------------------------------------------------
%	ORACLE AND BASELINE
%----------------------------------------------------------------------------------------

\section*{Oracle and Baseline} % Unnumbered section%

The Oracle will be a manual classification of articles, and impact score based on the content and its relevance to the subject.
Following is a table of the results of the articles that have any impact to the subject of "guns".
We are interested in articles that reflect a true impact to guns, in the context of gun control, gun legislation, NRA and 2nd amaendment
The following list of articles are categorized and evaluated.

Article Name    Category    Impact score(scale of 1 to 10)
53297           Politics    10
53294           Politics    10
176846          Politics    0
crime_report    Police      0

As per the Oracle, the scores are dependent on the context of the domain in question.
Hence even though the article "crime_report" has words like shooting and homicide, it doesnt have any impact on the domain of "guns".\newline

For the baseline we implemented a SGD ML classifier, where tfidf and vector count was used as features. With this we achieved an accuracy score of 82.4 percent. \newline

The output of the classifier is what we fed into the impact score analyser where we tried to rank the documents in terms of their impact score. \newline

For calculating the impact score we gave the function a set of words for a given category of the articles. Using these tokens we identified the frequency of these tokens in each of the articles. \newline

For our baseline we only tried to count the frequency of terms in the article from the set of tokens against which the impact score was being calculated.  However the numbers were not great.\newline

In order to improve the same we want to explore using the following features : \newline

a. Word2Vec : Using this we would include in the word frequency not just exact words which match but also words that are close in meaning. \newline
b. Wordvector averaging \newline
c. BERT and ELMO - Since these embedding take into account not just the word but also the context of the input this would help us improve how we calculate the impact score. \newline

We would also like to improve our classification model for which we want to look at using Neural Network (RNN using LSTM CNN) using Keras and Tensorflow. \newline

%---------i-------------------------------------------------------------------------------
%	CHALLENGES
%----------------------------------------------------------------------------------------

\section*{Challenges} % Unnumbered section

Following are some key challenges facing the project:
\begin{itemize}
	\item Build a system that understands impact score
	\item Build a classifier system that can compliment the impact score better
	\item Obtaining relevant data sets to train the system
\end{itemize}

 
%----------------------------------------------------------------------------------------
%	INFRASTRUCTURE
%----------------------------------------------------------------------------------------

\section*{Infrastructure} % Unnumbered section
Infrastructure for the project consists of building instrumentation for the model to run

\subsection*{Application}
The model will accept input from user on the category and will display the outputs as a stack rank of articles on the internet along with their potential impact scores. This application will interact with the model. This application will also be responsible for pulling articles from the internet in obtaining test data if required.

\subsection*{Compute}
Google cloud credits will be used to run this model and perform training of the model. 

\subsection*{Dataset}
We will be using the “20 Newsgroup” data set. The 20 Newsgroups data set is a collection of approximately 20,000 newsgroup documents, partitioned (nearly) evenly across 20 different newsgroups. To the best of my knowledge, it was originally collected by Ken Lang, probably for his Newsweeder: Learning to filter netnews paper, though he does not explicitly mention this collection. The 20 newsgroups collection has become a popular data set for experiments in text applications of machine learning techniques, such as text classification and text clustering. This comes builtin with scikit learn. And can be used directly from there. \newline
We are also exploring to use the bbc data set which consists of 2225 documents from the BBC news website corresponding to stories in five topical areas business, entertainment, politics, sport, tech. 

\maketitle % Print the title

%----------------------------------------------------------------------------------------
%	SOCIAL IMPACT
%----------------------------------------------------------------------------------------

\section*{Infrastructure} % Unnumbered section

\section*{Project Prompt}

\textit{Define the input-output behavior of the system and the scope of the project. What is your evaluation metric for success? Collect some preliminary data, and give concrete examples of inputs and outputs. Implement a baseline and an oracle and discuss the gap. What are the challenges? Which topics (e.g., search, MDPs, etc.) might be able to address those challenges (at a high-level, since we haven't covered any techniques in detail at this point)? Search the Internet for similar projects and mention the related work. You should basically have all the infrastructure (e.g., building a simulator, cleaning data) completed to do something interesting by now.}
\end{document}
